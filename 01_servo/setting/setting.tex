\documentclass[autodetect-engine,dvipdfmx-if-dvi,ja=standard]{bxjsarticle}

\usepackage{graphicx}        % 図を表示するのに必要
\usepackage{color}           % jpgなどを表示するのに必要
\usepackage{amsmath,amssymb} % 数学記号を出すのに必要
\usepackage{type1cm}         % fontsizeのエラー回避
\usepackage{here}            % 図の強制配置
\usepackage{url}             % URLをいい感じにしてくれる
\usepackage{subfigure}       % 図をまとめて表示
\usepackage{pdfpages}        % PDFの連結
\usepackage{setspace}
\usepackage{cases}
\usepackage{fancyhdr}


% 余白の設定
% \setlength{\textheight}{\paperheight}   % 紙面縦幅を本文領域にする(BOTTOM=-TOP)
% \setlength{\topmargin}{-15.4truemm}     % 上の余白を10mm(=1inch-15.4mm)に
% \addtolength{\topmargin}{-\headheight}  %
% \addtolength{\topmargin}{-\headsep}     % ヘッダの分だけ本文領域を移動させる
% \addtolength{\textheight}{-20truemm}    % 下の余白も10mm
% \setlength{\textwidth}{\paperwidth}     % 紙面横幅を本文領域にする(RIGHT=-LEFT)
% \setlength{\oddsidemargin}{-5.4truemm}  % 奇数ページの左の余白を20mm(=1inch-5.4mm)に
% \setlength{\evensidemargin}{-5.4truemm} % 偶数数ページの左の余白を20mm(=1inch-5.4mm)に
% \addtolength{\textwidth}{-40truemm}     % 右の余白も20mm

% タイトル
\title{タイトル}

% ヘッダとフッタの設定
% \lhead{電気電子情報工学実験}
% \chead{}
% \rhead{20315784 佐藤凌雅}
% \lfoot{}
% \cfoot{\thepage} % ページ数
% \rfoot{}

\parindent = 0pt  % 行頭の字下げをしない
\setstretch{1.0}  % 行間

% キャプションの英語化
\renewcommand{\figurename}{Fig.}
\renewcommand{\tablename}{Table}

% 各章,節などタイトルの大きさを変更
% \titleformat*{\section}{\Huge\bfseries}
% \titleformat*{\subsection}{\Large\bfseries}

% 式の番号を(senction_num.num)のようにする
\makeatletter
\@addtoreset{equation}{chapter}
\def\theequation{\thechapter.\arabic{equation}}
\makeatother

% 呼び出したページのページ番号を消す
\newcommand{\deletePageNum}{
    \thispagestyle{empty}
    \clearpage
    \addtocounter{page}{-1}
}

% urlのフォントを直す
\renewcommand\UrlFont{\rmfamily}
